%%=============================================================================
%% Samenvatting
%%=============================================================================

% TODO: De "abstract" of samenvatting is een kernachtige (~ 1 blz. voor een
% thesis) synthese van het document.
%
% Deze aspecten moeten zeker aan bod komen:
% - Context: waarom is dit werk belangrijk?
% - Nood: waarom moest dit onderzocht worden?
% - Taak: wat heb je precies gedaan?
% - Object: wat staat in dit document geschreven?
% - Resultaat: wat was het resultaat?
% - Conclusie: wat is/zijn de belangrijkste conclusie(s)?
% - Perspectief: blijven er nog vragen open die in de toekomst nog kunnen
%    onderzocht worden? Wat is een mogelijk vervolg voor jouw onderzoek?
%
% LET OP! Een samenvatting is GEEN voorwoord!

%%---------- Nederlandse samenvatting -----------------------------------------
%
% TODO: Als je je bachelorproef in het Engels schrijft, moet je eerst een
% Nederlandse samenvatting invoegen. Haal daarvoor onderstaande code uit
% commentaar.
% Wie zijn bachelorproef in het Nederlands schrijft, kan dit negeren, de inhoud
% wordt niet in het document ingevoegd.

\IfLanguageName{english}{%
\selectlanguage{dutch}
\chapter*{Samenvatting}

\selectlanguage{english}
}{}

%%---------- Samenvatting -----------------------------------------------------
% De samenvatting in de hoofdtaal van het document

\chapter*{\IfLanguageName{dutch}{Samenvatting}{Abstract}}

% CONTEXT

% NOOD
Dit onderwerp werd voorgesteld door delaware. Om te onderzoeken hoe microservice integration patterns een invloed kunnen hebben op het order-to-cash proces binnen SAP. 

% TAAK
Dit onderzoek kent een verdiepende structuur. Waar in de start uitleg wordt gegeven omtrent microservices, kent het vervolg een vergelijking tussen verschillende onderdelen. Enkele voorbeelden hiervan zijn authenticatie en authorisatie. Deze scriptie sluit af met opkomende ideologieën en de bescherming van microservices.

% OBJECT
In deze thesis is als eerste een uitgebreide literatuurstudie terug te vinden. In deze literatuurstudie wordt er dieper ingegaan op de technologie van microservice en de verschillende integration patterns omtrent de microservices. Daarnaast zal er meer uitleg gegeven worden over het order-to-cash proces.
Na de literatuurstudie wordt het microservice integration pattern toegepast op het order-to-cash proces. Als afsluiter komt de conclusie.

% RESULTAAT
Microservices zorgen voor een geautomatiseerd proces met weinig interactie met een werknemer. Het is een complexe architectuur. De overschakeling bevat veel onderdelen waar er in eerste instantie niet aan gedacht wordt. Het werd ook duidelijk dat er goed moet worden nagedacht over alle kleine dingen, zoals authenticatie, de manier van bescherming, hoe logging wordt opgevangen. 

% CONCLUSIE
De overschakeling kan gemakkelijk onderschat worden.
Een belangrijke vraag die men kan hebben is: hebben microservices wel een meerwaarde in dit proces? Of de overschakeling moet gemaakt worden, is afhankelijk van wat de noden de applicatie precies zijn.

% PERSPECTIEF
Er zullen zeker vragen en aanvullingen komen in de toekomst. Dit is een evoluerende technologie en er zullen betere oplossingen komen voor onderdelen.